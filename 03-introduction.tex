\chapter*{ВВЕДЕНИЕ}
\addcontentsline{toc}{chapter}{ВВЕДЕНИЕ}

Морфинг трехмерных моделей представляет собой технику анимации в компьютерной графике, направленную на плавное переходное изменение одной трехмерной модели в другую. 
Этот процесс создает эффект плавного метаморфоза между двумя или более формами, что может быть использовано для создания динамичных и визуально привлекательных анимаций.

Морфинг может применяться в различных областях, таких как компьютерная анимация, визуализация данных, медицинская графика и игровая индустрия. 
Эта техника позволяет создавать плавные и естественные анимации, предоставляя художникам и разработчикам инструмент для трансформации объектов внутри виртуальных сцен.

Целью данной работы является разработка программного обеспечения для морфинга трехмерных моделей. Для достижения поставленной цели требуется решить следующие задачи.
\begin{enumerate}
	\item Формализовать представление объектов и описать их.
	\item Проанализировать алгоритмы морфинга трехмерных моделей и выбрать наилучшие для достижения цели.
	\item Выбрать средства реализации алгоритмов.
	\item Реализовать выбранные алгоритмы.
	\item Реализовать графический интерфейс.
	\item Исследовать временные характеристики выбранных алгоритмов на основе созданного программного обеспечения.
\end{enumerate}
